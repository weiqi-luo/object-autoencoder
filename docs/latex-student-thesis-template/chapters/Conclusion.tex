%_____Zusammenfassung, Ausblick_________________________________
\chapter{Conclusion}
In this project, I implemented the pose estimation approach proposed in \cite{sundermeyer2018implicit} and tested it in different scenarios. The evaluation result shows that this approach is able to handle the pose ambiguity problem due to the symmetrical structure of objects. Furthermore, it shows a good generalization capability in different environment simulated by dataset augmentation. 
\\[8pt]
However, due to constraints of time and computing power, only the 3D orientation of the object pose is estimated in my project. In the future, the project can be extended to estimate the 6D pose of the object. And stronger augmentation, like occlusions with random object masks may be further applied to the training dataset in order to strengthen the feature extraction capability of the autoencoder against occlusion. 